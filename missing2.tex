\documentclass{ctexart}

\usepackage{graphicx}
\usepackage{float}
\usepackage{hyperref}

%\usepackage[section]{placeins}




\usepackage{zhnumber} % change section number to chinese
\renewcommand\thesection{\zhnum{section}}
\renewcommand \thesubsection {\arabic{section}}



\begin{document}


\title{中国海洋大学\\系统开发工具基础实验报告}
\author{学生姓名:兰春光\hspace{20}学号:23020007053\hspace{20}指导老师:周小伟}
\date{\hspace{5}实验时间:\today}
\maketitle
\footnotesize\tableofcontents




\section{实验名称:shell工具和脚本,Vim,数据整理}

github 链接:\href{URL}{https://github.com/lan200503/System-Development-Tools-}


\section{实验实例(内容和结果)}

\subsection{shell工具和脚本}

\subsubsection{使用 echo \$SHELL 命令可以查看shell}
\begin{figure}[H]
    \centering
    \includegraphics[width=0.5\linewidth]{echoSHELL.png}
    \caption{echo \$shell}
    \label{fig:enter-label}
\end{figure}
可以看到当前使用的是bash

\subsubsection{新建一个名为 missing 的文件夹}
\begin{figure}[H]
    \centering
    \includegraphics[width=0.5\linewidth]{image.png}
    \caption{mkdir missing}
    \label{fig:enter-label}
\end{figure}

\subsubsection{用 touch 在 missing 文件夹中新建一个叫 semester 的文件}

\begin{figure}[H]
    \centering
    \includegraphics[width=0.5\linewidth]{touch.png}
    \caption{touch semester}
    \label{fig:enter-label}
\end{figure}

\subsubsection{将以下内容一行一行地写入 semester 文件:}
 \#!/bin/sh \\
 curl --head --silent https://missing.csail.mit.edu \\
\begin{figure}[H]
    \centering
    \includegraphics[width=0.5\linewidth]{echo1.png}
    \caption{echo '' >> <file>}
    \label{fig:enter-label}
\end{figure}

\subsubsection{尝试执行这个文件。}
\begin{figure}[h]
    \centering
    \includegraphics[width=0.5\linewidth]{semester.png}
    \caption{./semester}
    \label{fig:enter-label}
\end{figure}
git bash 和 Linux 的结果不太一样. Linux 会显示权限不足

\subsubsection{查看 chmod 的手册}
\begin{figure}[H]
    \centering
    \includegraphics[width=0.5\linewidth]{chmod --help.png}
    \caption{chmod --help}
    \label{fig:enter-label}
\end{figure}

\subsubsection{使用 chmod 命令改变权限,使 ./semester 能够成功执行,不要使用 sh semester 来执行该程序。}

\begin{figure}[h]
    \centering
    \includegraphics[width=0.5\linewidth]{chmod +x.png}
    \caption{chmod +x}
    \label{fig:enter-label}
\end{figure}

\subsubsection{使用 | 和 > ,将 semester 文件输出的最后更改日期信息,写入主目录下的 last-modified.txt 的文件中}

\begin{figure}[H]
    \centering
    \includegraphics[width=0.5\linewidth]{semester.png}
    \caption{./semester | grep "last-modified" > ~/last-modified.txt}
    \label{fig:enter-label}
\end{figure}


\subsubsection{ 命令:ls -ahtl --color=auto ,可实现所有文件(包括隐藏文件)、以人类可以理解的格式输出、以最近访问顺序排序、以彩色文本显示输出结果:}

\begin{figure}[H]
    \centering
    \includegraphics[width=0.5\linewidth]{ls.png}
    \caption{ls -ahtl --color=auto}
    \label{fig:enter-label}
\end{figure}

\subsubsection{编写两个bash函数 marco 和 polo 执行下面的操作。}
每当你执行 marco 时,当前的工作目录应当以某种形式保存,当执行 last polo 时,无论现
在处在什么目录下,都应当 cd 回到当时执行 marco 的目录。为了方便debug,你可以把代码写在单独的文件marco.sh 中,并通过 source marco.sh 命令,(重新)加载函数。
\begin{figure}[H]
    \centering
    \includegraphics[width=0.5\linewidth]{vim.png}
    \caption{vim编辑}
    \label{fig:enter-label}
\end{figure}

\begin{figure}[H]
    \centering
    \includegraphics[width=0.5\linewidth]{source marco.sh.png}
    \caption{source marco.sh}
    \label{fig:enter-label}
\end{figure}


\subsubsection{编写一段bash脚本,运行如下的脚本直到它出错,将它的标
准输出和标准错误流记录到文件,并在最后输出所有内容。}
\#!/usr/bin/env bash \\
n=\$(( RANDOM \% 100 )) \\
if [[ n -eq 42 ]]; then \\
    echo "Something went wrong" \\
    >&2 echo "The error was using magic numbers" \\
    exit 1\\
fi\\
    echo "Everything went according to plan" \\
将该脚本写入3.sh 中\\
 \begin{figure}[h]
     \centering
     \includegraphics[width=0.5\linewidth]{3-debug.png}
     \caption{3-debug}
     \label{fig:enter-label}
 \end{figure}
 
 \begin{figure}[h]
     \centering
     \includegraphics[width=0.5\linewidth]{output.png}
     \caption{output}
     \label{fig:enter-label}
 \end{figure}

\subsubsection{按照最近的使用时间列出文件的命令:find . -type f -print | xargs -d '\n' ls -t}
\begin{figure}[H]
    \centering
    \includegraphics[width=0.5\linewidth]{find.png}
    \caption{find . -type f -print | xargs -d '\n' ls -t}
    \label{fig:enter-label}
\end{figure}

\subsection{Vim}
\subsubsection{vim 基本操作}
普通模式:这是 vim 的默认模式,用于输入命令。\\
插入模式:用于输入文本。\\
命令行模式:用于输入命令,如保存文件、退出等。\\
x:删除光标所在位置的字符\\
dd:删除整行\\
d\$:删除从光标位置到行尾的内容\\
:w 保存\\
:q 退出\\
:wq 保存并退出\\
\subsubsection{替换}
\begin{figure}[h]
    \centering
    \includegraphics[width=0.5\linewidth]{替换.png}
    \caption{:\%s/文件2/文件22/g}
    \label{fig:enter-label}
\end{figure}


\subsection{数据处理}


\subsubsection{mkdir}
\begin{figure}[h]
    \centering
    \includegraphics[width=0.5\linewidth]{mkdir.png}
    \caption{mkdir missing-semester}
    \label{fig:enter-label}
\end{figure}

\subsubsection{cat}
\begin{figure}[h]
    \centering
    \includegraphics[width=0.5\linewidth]{cat.png}
    \caption{cat file2.txt}
    \label{fig:enter-label}
\end{figure}

\subsubsection{grep}
\begin{figure}[h]
    \centering
    \includegraphics[width=0.5\linewidth]{grep.png}
    \caption{grep "文件" file2.txt}
    \label{fig:enter-label}
\end{figure}

\subsubsection{awk}
\begin{figure}[h]
    \centering
    \includegraphics[width=0.5\linewidth]{awk.png}
    \caption{awk 'NR==2' file2.txt}
    \label{fig:enter-label}
\end{figure}


\subsubsection{sed}
\begin{figure}[h]
    \centering
    \includegraphics[width=0.5\linewidth]{sed.png}
    \caption{sed 's/old/new/g' file2.txt > file1.txt}
    \label{fig:enter-label}
\end{figure}

\subsubsection{rm}
\begin{figure}[H]
    \centering
    \includegraphics[width=0.5\linewidth]{rm.png}
    \caption{rm file1.txt}
    \label{fig:enter-label}
\end{figure}

\subsubsection{mv}
\begin{figure}[H]
    \centering
    \includegraphics[width=0.5\linewidth]{mv.png}
    \caption{mv file2.txt file22.txt}
    \label{fig:enter-label}
\end{figure}


\section{解题感悟}
学习 Shell 工具和脚本、Vim 编辑器以及数据整理,让我深刻体会到了命令行的强大和灵活性。Shell 脚本自动化处理任务的能力极大地提高了工作效率。 Vim 作为高效的文本编辑器,在代码编辑时更加灵活便捷。数据整理则让我更加熟悉对文件等一系列操作。希望以后能系统性地学习Linux 系统。


\end{document}
