\documentclass{ctexart}

\usepackage{graphicx}
\usepackage{float}
\usepackage{hyperref}

%\usepackage[section]{placeins}




\usepackage{zhnumber} % change section number to chinese
\renewcommand\thesection{\zhnum{section}}
\renewcommand \thesubsection {\arabic{section}}



\begin{document}


\title{中国海洋大学\\系统开发工具基础实验报告}
\author{学生姓名:兰春光\hspace{20}学号:23020007053\hspace{20}指导老师:周小伟}
\date{\hspace{5}实验时间:\today}
\maketitle
\footnotesize\tableofcontents




\section{实验名称:版本控制,\LaTeX 文档编辑}

github 链接:\href{URL}{https://github.com/lan200503/System-Development-Tools-}

\section{实验实例(内容和结果)}

\subsection{版本控制}

\subsubsection{git init}
创建本地仓库\\
\begin{figure}[h]
    
    \centering
    \includegraphics[width=0.5\linewidth]{git init.png}
    \caption{git init}
    \label{fig:enter-label}
\end{figure}

\subsubsection{git clone}
克隆本地仓库\\
\begin{figure}[h]
    \centering
    \includegraphics[width=0.5\linewidth]{git clone.png}
    \caption{git clone}
    \label{fig:enter-label}
\end{figure}

\subsubsection{git add}
创建文件并添加文件到暂存区\\
\begin{figure}[H] 
    
    \centering
    \includegraphics[width=0.5\linewidth]{git add.png}
    \caption{git add}
    \label{fig:enter-label}
\end{figure}

\subsubsection{git status}
查看仓库状态\\
\begin{figure}[h]    
    \centering
    \includegraphics[width=0.5\linewidth]{git status.png}
    \caption{git status}
    \label{fig:enter-label}
    %\FloatBarrier
\end{figure}

\subsubsection{git reset}
回退仓库版本\\
\begin{figure}[h]
    \centering
    \includegraphics[width=0.5\linewidth]{git reset.png}
    \caption{git reset}
    \label{fig:enter-label}
\end{figure}

\subsubsection{git commit}
提交暂存区的文件到本地仓库
\begin{figure}[h]
    \centering
    \includegraphics[width=0.5\linewidth]{git commit.png}
    \caption{git commit}
    \label{fig:enter-label}
\end{figure}


\subsubsection{git log}
查看仓库的提交记录
\begin{figure}[h]
    \centering
    \includegraphics[width=0.5\linewidth]{git log.png}
    \caption{git log}
    \label{fig:enter-label}
\end{figure}

\subsubsection{git diff}
默认查看工作区和暂存区的区别
\begin{figure}[h]
    \centering
    \includegraphics[width=0.5\linewidth]{git diff.png}
    \caption{git diff}
    \label{fig:enter-label}
\end{figure}

\subsubsection{git rm}
删除暂存区和工作区的文件
\begin{figure}[H]
    \centering
    \includegraphics[width=0.5\linewidth]{git rm.png}
    \caption{git rm}
    \label{fig:enter-label}
\end{figure}

\subsubsection{git ls-files}
查看缓存区的所有文件
\begin{figure}[h]
    \centering
    \includegraphics[width=0.5\linewidth]{git ls-files.png}
    \caption{git ls-files}
    \label{fig:enter-label}
\end{figure}

\subsection{版本控制 习题}
\begin{figure}[h]
    \centering
    \includegraphics[width=0.5\linewidth]{习题.png}
    \caption{习题}
    \label{fig:enter-label}
\end{figure}
\subsubsection{可视化}
\begin{figure}[h]
    \centering
    \includegraphics[width=0.5\linewidth]{可视化.png}
    \caption{git log --all --graph --decorate}
    \label{fig:enter-label}
\end{figure}
\subsubsection{谁最后修改了README.md}
\begin{figure}[h]
    \centering
    \includegraphics[width=0.5\linewidth]{修改README.png}
    \caption{git log README.md}
    \label{fig:enter-label}
\end{figure}
\subsubsection{修改某个文件的提交信息}
\begin{figure}[h]
    \centering
    \includegraphics[width=0.5\linewidth]{git blame.png}
    \caption{git blame _config.yml}
    \label{fig:enter-label}
\end{figure}



\subsection{\LaTeX 文档编辑}
\subsubsection{documentclass}
设置文档类型
\begin{figure}[h]
    \centering
    \includegraphics[width=0.5\linewidth]{documentclass.png}
    \caption{documentclass}
    \label{fig:enter-label}
\end{figure}

\subsubsection{usepackage}
调用宏包\\
\begin{figure}[H]
    \centering
    \includegraphics[width=0.5\linewidth]{usepackage{ctex}.png}
    \caption{usepackage{ctex}}
    \label{fig:enter-label}
\end{figure}

\subsubsection{section}
设置标题,副标题\\
\begin{figure}[H]
    \centering
    \includegraphics[width=0.5\linewidth]{section.png}
    \caption{section}
    \label{fig:enter-label}
\end{figure}

\subsubsection{title,author,date}
设置标题作者日期,制作封面\\
\begin{figure}[h]
    \centering
    \includegraphics[width=0.5\linewidth]{title.png}
    \caption{title}
    \label{fig:enter-label}
\end{figure}

\subsubsection{equation}
插入公式并用了希腊字母\\
\begin{figure}[h]
    \centering
    \includegraphics[width=0.5\linewidth]{equation.png}
    \caption{equation}
    \label{fig:enter-label}
\end{figure}

\subsubsection{graphics}
插入图片\\
\begin{figure}[h]
    \centering
    \includegraphics[width=0.5\linewidth]{graphics.png}
    \caption{graphics}
    \label{fig:enter-label}
\end{figure}

\subsubsection{label,ref}
设置标签和引用\\
\begin{figure}[h]
    \centering
    \includegraphics[width=0.5\linewidth]{lable.png}
    \caption{lable}
    \label{fig:enter-label}
\end{figure}

\subsubsection{tableofcontents}
生成目录,跟section等有关\\
\begin{figure}[h]
    \centering
    \includegraphics[width=0.5\linewidth]{tableofcontents.png}
    \caption{tableofcontents}
    \label{fig:enter-label}
\end{figure}

\subsubsection{font effects}
各种字体效果\\
\begin{figure}[h]
    \centering
    \includegraphics[width=0.5\linewidth]{font effects.png}
    \caption{font effects}
    \label{fig:enter-label}
\end{figure}

\subsubsection{size}
各种字体大小,但不限于字体,可以用于目录\\
\begin{figure}[h]
    \centering
    \includegraphics[width=0.5\linewidth]{size.png}
    \caption{size}
    \label{fig:enter-label}
\end{figure}

\section{解题感悟}
通过使用git创建本地仓库,并与github上的远程仓库进行链接,让我感受到了git的强大以及git的实用程度,我要持续跟进学习,尽量掌握git常用指令。\\ 
通过使用latex编辑文档,了解到latex在某些方面编辑文档的便捷性,掌握latex的语法十分必要,对以后编辑文档十分受益。



\end{document}
